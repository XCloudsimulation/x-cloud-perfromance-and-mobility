\section{Introduction}

Mobile services and \ac{MD}es \footnote{Any user client device accessing a service, such as a mobile phone} functions are at an increasing rate being virtualized and augmented to the cloud. Rich Mobile Applications \cite{March2011618} will soon, more often than not, be seamlessly executed, either partially or fully in the cloud. Alongside applications, fundamental \ac{MD} resources, such as storage and CPU, are being augmented to the cloud. In this resource paradigm, the border between what is being executed locally and remotely is blurred as developers are given more powerful tools to tap into remote ubiquitous generic virtual resources. Additionally, the advent of the internet of things will contribute with a vast number of new types of wireless devices, actuators, and sensors querying and connecting to remote cloud resources. 

% Rich mobile applications
As we begin to rely more on remote ubiquitous resources we also grow more dependant on the quality of the intermediate WAN network and by the geographical separation of the \ac{MD} and the \ac{DC} \cite{choi2007analysis}. Latency sensitive applications and cognitive augmentation services, such as process controls, latency sensitive storage, real time video game rendering, and augmented reality video analysis will quickly falter if subject to a  communication delay.

Virtual resources are accessed through increasingly congested mobile access networks, as more devices are crowding the mobile networks and applications are generating and receiving more data. This congestion contributes to unwanted communication latency \cite{hu2005measurement}. In addition to congestion, the geographic discrepancy between the \ac{MD} and the \ac{DC} introduces a propagation delay, bounded by the speed of light.

The \xcloud{} paradigm, put forward by \cite{chandra2013decentralized,ericsson_akami,satyanarayanan2009case,kiukkonen2010towards,March2011618}, attempts to remedy the aforementioned congestion and latency performance inhibitors by locating cloud resources at the edged of, and adjacent to, the mobile access network. In the ad-hoc scenario, resources are shared amongst \ac{MD} where each connected \ac{MD} surrenders its available resources to its peers. In its centralized form, \ac{DC} resources are proposedly located at the edge of the network, adjacent or integrated into an \ac{RBS}, catering for the \ac{MD}s located within its cell coverage. Alternatively, or complimentary, \ac{DC}s are proposedly integrated with resources in the common administrative nodes of the proposed virtualized radio access networks. The scale and degree of dispersion can be optimized for each application, given the applications resource tiers and its users mobility behaviour.

Round trip time, is arguably proportional to the geographic distance between the \ac{MD} and the \ac{DC}. Services hosted in the \xcloud{} are migrated with the \ac{MD}, through the network, to minimize this incurred latency and congestion on the adjacent WAN. In practice, services, or rather the VMs that host the services, are migrated to a \ac{DC} that, is available, provides the lowest service latency, and incurs least global network congestion. Doing so might minimize the experienced delay for the \ac{MD}, but will incur a migration overhead in the hosting \ac{DC} and in the network over which the VM is migrated or duplicated. Conceivably, various provisioning schemes and cost functions can be deployed to minimize both the delay experienced by the user and the added resource strain on the \ac{DC} and the intermediate network.

\ac{MD} mobility is a key differentiator between traditional cloud computing with distant \ac{DC}s and the \xcloud{}, and is a fundamental dynamic property of a \xcloud{}. In order to be able to optimize the \xcloud{} topology, it is essential to understand how \ac{MD} mobility affects the perceived service performance and what load it imposes on the network.

The topology paradigms of tomorrows all-IP (Internet Protocol) mobile networks \cite{6144211,5357099} are hot topics of research, but one can assume that they will be influenced by the notion of virtualized resources \cite{baroncelli2010network, chowdhury2009network}. Large portions of \ac{RBS}s can proposedly be virtualized and centralized to a common \ac{DC} with a locally-bounded service domain, shared amongst several \ac{RBS}s, leaving the \ac{RBS}s, in principal, with just the radio interface \cite{melzercloud}. The degree of centralization is conceivably geographically bounded by propagation delay and signal attenuation, and is resource hampered by the aggregated traffic that passes through the dedicated \ac{DC}. There is to our knowledge, very little research exploring future mobile Telecom infrastructure topologies with the \xcloud{} in mind. There is on the other hand, extensive research directed at exploring relevant economic and IT models of how to integrate existing Telecom services to the cloud and how to apply Telecom-grade SLAs to existing cloud services \cite{EricssonWhitePaper,6156350, 5357099}. These services are frequently proposed to reside in the network and be managed by the Telecom operators.

The concept of geo-distributed cloud resources has received some research attention over the past few years, but has had a clear research focus on storage and shared data. The authors of \cite{agarwal2010volley} present a method to geographically migrate shared data resources globally, not only to minimize the distance between the \ac{MD} and the \ac{DC}, and thus service latency, but also to globally load-balance the hosting \ac{DC} on which the observed service is distributively hosted. Their results reveal a significant reduction in service latency, inter-\ac{DC} communication, and contributed WAN congestion. Their proposed control process runs over long time periods and operate on a global scale with relatively geographically static users. Although sharing some fundamental dynamics, albeit at different scales, in contrast, the \xcloud{} paradigm, \ac{MD} movement between gerographic \ac{DC} domaons is proportionally more rapid. Additionally, from a network perspective, hand-overs between radio base stations is likely to occur during a service session. Additionally, \xcloud{} virtualized resources are assumed to be universal and do not just include storage, they vary in size and capabilities,are deployed by the Telecom operators, and are based on local needs and demand.

The field of \xcloud{} has much in common with field of geo-distributed cloud resources, but is dominated by the notions of augmenting \ac{MD}s through virtualizing their resources \cite{6563280} and reducing service response times through geo-cascaded data caching \cite{1437087,ericsson_akami}. As a result, much of the research is concerned with coping with specific dynamics, and do thus not address the generic case of small geo-distributed \ac{DC}s, serving a local mobile subscriber populous. There are, to our knowledge, significant research gaps in how cloud services perform when hyper-dispersed and rapidly migrated. Additionally, there is little research on how the \xcloud{} can be accommodated in and optimized for future network topologies.

In this paper we investigate the fundamental effects of \ac{MD} mobility on the \xcloud{} by observing \ac{DC} utilization, the proportion of \ac{DC}s resources spent on migrating services, and how service performance is affected by migration. In addition, we propose a simulation model built around the fundamental dynamics that contribute to package latency and VM utilization, and is designed to examine the fundamental and generic resource problems in a \xcloud{} of mobile \ac{MD}s. The models include a generic mobile network, populated with \ac{MD}s subscribing to a number of services, served by a number of locally geo-distributed \ac{DC}s. The simulation model is subjected to multiple scenarios in constellations of varying number of users, services, and \ac{DC} clustering.

In addition to compositions that include NS-3, there is a selection of simularly fashioned, purpose built simulators, with a finer level of detail, such as \cite{meisner2012bighouse,calheiros2011cloudsim}. However, none of them operat at the abstarction level we desire and do not model mobile networks, data centers, and migration of data.

The simulated scenarios reveal ...
%This paper provides an investigation into the fundamental effects of \ue{} in the \xcloud{} in relation to the number of subscribers, the abstract placement of the \dc{}, and the number of services. %An optimal or reasonable technical bounds for the \xcloud{} topology is not yet to be determined. This paper disregards the deeper technical and topological constrains of existing mobile systems in order to provide fundamental results that can be employed to shape the forthcoming mobile network generations.

In this paper, Section \ref{sec:desiard_model} details which aspects and abstractions of the \xcloud{} topology that are included in our experiments. Furthermore, the simulation model is specified in Section \ref{sec:simulation_model}. Section \ref{sec:experiments} details the specifics of the simulation experiments. Lastly, Sections \ref{sec:results} and \ref{sec:conclusions} present the results and the consultations drawn from the experiments.


%What to add 
%Proposed model
