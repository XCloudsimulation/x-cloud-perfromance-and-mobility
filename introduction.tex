\section{Introduction}
\Ue{} mobility is a key differentiator between traditional cloud computing with distant \dcs{} and the \xcloud{}, and is a fundamental dynamic property of a \xcloud{}. It is therefore essential to understand how \ue{} mobility affects the perceived service performance and what load it imposes on the network in the generic case.

Mobile services and \ue{} \footnote{Any user client device accesing the service, such as a mobile phone} functions are at an increasing rate being virtualized and augmented to the cloud. Applications will soon more often than not be seamlessly executed, partially or fully in the cloud. Alongside applications fundamental \ue{} resources, such as storage and CPU, are being virtualized to the cloud. In this paradigm, the border between what is being executed locally and remotely is blurred as developers are given more powerful tools to tap into remote ubiquitous generic virtual resources. This resource paradigm, has overwhelmingly augmented the capabilities of mobile applications, and enabled collaborative computing. In the years to come, at the dawn of the era of the Internet of things, just short of all devices will contribute data to the cloud and/or utilize its resources.

% Rich mobile applications

As we begin to rely more on remote resources we also grow more suceptible to the communication delay introduced by the the intermediate WAN network and by the geographical separation of the \ue{} and the \dc{} \cite{choi2007analysis}. Latency sensitive applications such as process controls, latency sensitive storage, real time video game rendering, and augmented reality video analysis will quickly falter if subject to a significant and varying communication delay.

Virtual resources are accessed through increasingly congested mobile access networks. More devices are crowding the mobile networks and applications are generating and receiving more data, this congestion translates into delay or latency \cite{hu2005measurement}. Additionally, the geographic distance to the data centre introduces a propagation delay, bounded by the speed of light.

The \xcloud{} paradigm, put forward by \cite{chandra2013decentralized,ericsson_akami,satyanarayanan2009case,kiukkonen2010towards}, attemps to remedy the aforementioned congestion and latency performance inhibitors by locating cloud resources at the edged of and adjacent to the mobile access networks. In the ad-hoc scenario, resources are shared amongst \ues{} as each \ue{} surrenders its available resources generically to its peers. However, from a network perspective, at one extreme, \dc{} resources can proposedly be located at the edge of the network, adjacent or integrated into an \rbs{}, catering for the \ues{} located within its cell coverage. Alternatively, or complimentary, \dcs{} can be integrated with resources in the proposed forthcoming virtualized radio access networks. The scale and the degree of dispersion can be optimized for each application, given the applications resource tiers and its users mobility behaviour.

Round trip time, is proportioanl to the geographic proximity between the \ue{} and the \dc{}. To that effect, services hosted in the \xcloud{} are migrated with the \ue{}, through the network, to minimize this incurred latency. In practice, services, or rather the VMs that host the services, will be migrated to the \dc{} that, is available, provides the lowest service latency, and incurs least global network congestion. Doing so might minimize the experienced service delay for the \ue{}, but will incurr a migration overhead in the hosting \dc{} and in the network over which the VM is migrated. Conceivably, various schemes and cost functions can be deployed to minimize both the delay experienced by the user and the added resource strain to the \dc{} and the network.

The topology paradigms of tomorrows all-IP (Internet Protocol) mobile networks \cite{6144211,5357099} are still to be determined, but one can assume that they will be influenced by the notion of virtualized resources \cite{baroncelli2010network, chowdhury2009network}. Large portions of \rbss{} can proposedly be virtualized and centralized to a common \dc with a localy-bounded service domain, shared by several \rbss{}, leaving the \rbss, in principal, with just the radio interface \cite{melzercloud}. The expanse of the centralization is geographically bounded by propagation delay and signal attenuation, and is resource hampered by the aggregated traffic that passes through the dedicated \dc. There is to our knowledge, very little research exploring future mobile telecom infrasturcture topologies with the \xcloud{} in mind. There is on the other hand, extensive research directed at exploring relevant economic and IT models of how to integrate existing telecom servies to the cloud and how to apply telecom-grade SLAs to existing cloud services \cite{EricssonWhitePaper,6156350, 5357099}.

The concept of geo-distributed cloud resources has been worked on for a few years, but with a clear focus on storage and shared data. The authors of \cite{agarwal2010volley} present a method to geographically migrate shared data resources globally, not only to minimize the distance between the \ue{} and the \dc{}, and thus service latency, but also to globally load-balance the hosting \dcs{}. Their results reveal a significant reduction in service latency, inter-\dc{} communication, and contributed WAN congestion. Their proposed control process runs over longer periods of time and operate on a global scale with geographically static users. Although sharing some fundamental dynamics, albeit at different scales, in contrast, in the \xcloud{} paradigm, \ue{} movement is much more rapid and proportional to the size of a session. Additionally, \xcloud{} virtualized resources are assumed to be universal and do not just include data and vary in size and capabilities.

The field of \xcloud{} has much in common with field of geo-distributed cloud resources, but is dominated by the notions of augmenting \ues{} through virtualizing their resources \cite{6563280} and reducing service response times through geo-cascaded data caching \cite{1437087,ericsson_akami}. As a result, much of the research is concerned with coping with specific dynamics, and do thus not address the generic case of small geo-distributed \dcs, serving a local mobile subscriber populous. There is large amount of work left to explore the fundamental dynamics of the \xcloud{} in order to be able to consider specific applications and use-cases.

This paper contributes with models designed to examine the fundamental and generic resource problems in a \xcloud{} of mobile \ues{}. The models include a generic mobile network inhabited by \ues{} subscribing to a number of services, served by a number of locally geo-distributed \dcs{}.

This paper provides an investigation into the fundamental effects of \ue{} in the \xcloud{} in relation to the number of subscribers, the abstract placement of the \dc{}, and the number of services. %An optimal or reasonable technical bounds for the \xcloud{} topology is not yet to be determined. This paper disregards the deeper technical and topological constrains of existing mobile systems in order to provide fundamental results that can be employed to shape the forthcoming mobile network generations.

%What to add 
%Proposed model
