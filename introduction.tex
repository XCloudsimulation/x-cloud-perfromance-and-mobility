\section{Introduction}
Mobile services and \ue functions are at an increasing rate beeing virtualized and augmented to the cloud. Applications are soon more ofthen than not seamliessly executed, partially or fully in the cloud. Alongside applications, fundemental \ue resources, such as storage and CPU, are being virtualized to the cloud. In this paradigm, the border between what is beeing executed locally and remotely is blurred as developers are given more powerful tools to tap into remote ubiquitous generic virtual resources. This resource paradigm, has overwhelmingly augmented the capabilities of mobile applications, and enabled collaborateive computing. In the years to come, just short of all devices will contribute data to the cloud and\/or utilize its resources.

As we begin to rely more on remote resources we also grow more dependant on the communication delay introduced by the the intermediate WAN network and by the geographical separation of the \ue and the \dc. Latency sensitive applications such as process controlls, latency sensitive storage, real time video game rendering, and augmented reality video analysis will quickly faulter if subject to a significant and varying communication delay.

The virtual resources are accessed through increasingly congested mobile access networks. More devices are crowding the mobile networks and applications generating and receiving more data, this congestion translates into delay. Additionally, the geographic distance to the data centre introduces a propagation delay, bounded by the speed of light.

The \xcloud paradigm, put forward by \cite{chandra2013decentralized,ericsson_akami}, attemps to remedy the aformentioned congestion and latency by locating cloud resources at the edged of and adjacent to the mobile access networks. In the ad-hoc scenario, resources are shared amongst \ues as each \ue surrenders its availbale resources generically to its peers. However, from a network perspective, at one extreme \dc resources can propsedly be located in at the edge of the network, adjacent or integrated into an \rbs, catering for the \ues reciding within its cell. Alternatively, or complimenarty, \dcs can be integrated with resources in the proposed forthcoming virtualized radio access networks. The scale and the degree of dispersion can be optimized for each application, given the applications resource tiers and its users mobility behaviour.

The geograpghic proximity between the \ue and the \dc is proportional to application service delay, to that effect, services hosted in the \xcloud are migrated with the \ue, through the network, to minimize this incurred latency. In practice, services, or rather the VMs that host the services will be migrated to the node that is available, provides the lowest delay, and incurrs least global network congestion. However, by doing so might minimize the experiance delay for the \ue, but will incurr a migration overhead in \dc and in the network a VM is migrated. Conceivably, various schemes and cost functions can be deployed to minimize both the delay experianced by the user and the added resource strain to th \dc and the network.

The topology paradigms of tomorrows all-ap mobile networks all-IP (Internet Protocol) \cite{6144211,5357099} are yet to be determined, but one can assume that they will be influenced by the notion of virtualized resources \cite{baroncelli2010network, chowdhury2009network}. Large portions of \rbss can proposedly be virtualized and centrallized to a common local-geographic \dc, shared by several \rbss, leaving the \rbss, in principal, with just the radio interface \cite{melzercloud}. The expanse of the centralization is geographically bounded by propagation delay and signal attenudation, and is resource hampered by the aggregate traffic that passes through the dedicated \dc. There is extensive research directed at exploring relevant economic and IT models \cite{EricssonWhitePaper,6156350, 5357099}.

The concept of geo-disributed cloud resources has been worked on for a few years, but with a clear focus on storage and sharded data. The authors of \cite{agarwal2010volley} pressent a method to geographically migrate shared data resouces globaly, not only to minimize the distance between the \ue and the \dc, and thus service latency, but also to globaly load-balance the hosting \dcs. Their results reveal a significant reduction in service latency, inter-\dc communication, and contributed WAN congestion. Their proposed controll process runs over longer periods of time and operate on a global scale with georgraphically static users. Although sharing some fundamental dynamisc, ableit at different scales, in contrast, in the \xcloud paradigm, \ue movement is much more rapid and proportional to the size of a session. Additinally, \xcloud virtualized resources are assumed to be universal and do not just cover data, and vary in size and capabilities.

The field of \xcloud bears much in common with geo-disributed cloud resources but is dominated by the notions of augmenting \ues through virtualizing their resources \cite{6563280} and reducing service response times through geo-cascaded data caching \cite{1437087,ericsson_akami}. As a result, much of the research is concerned with coping with specific dynamics, and do thus not address the generic case of generic locally geo-distributed resouces serving a local subscriber populous. There is large amount of work left to explore the fundamental dynamics of the \xcloud in order to be able to begin to consider specific applications and use-cases.

\Ue mobility is a key differentiator between traditional distant immobile clouds and the \xcloud, and is a fundamental dynamic property of an \xcloud. It is therefore essential to understand how \ue mobility affects the perceived service performance and what load it imposes on the network in the generic case.

This paper contributes with models designed to examine the fundamental and generic resource problems in an \xcloud of mobile \ues. The models include a generic mobile network inhabited by \ues subscribing to a number of services, served by a number of locally geo-distributed \dcs.

This paper provides an investigation into the fundamental effects of user \ue in the \xcloud in relation to the number of subscribers, the abstract placement of the servers, and the numer of services. An optimal or reasonable technical bounds for the \xcloud topology is not yet to be determined. This paper disregards the deeper techincal and topological constrains of existing mobile systems in order to provide fundamental resulst that can be employed to shape the forthcoming mobile nework generations.