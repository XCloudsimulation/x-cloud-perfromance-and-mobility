\section{Introduction}
Mobile service and function are at an increasing rate beeing virtualized and augmented to the cloud. Applications are soon more ofthen than not seamliessly executed, partially or fully in the cloud. Alongside applications, fundemental \ue resources, such as storage and CPU, are being virtualized. In this paradigm, the border between what is beeing executed locally and remotely is blurred as developers are given more powerful tools to tap into remote ubiquitous generic virtual resources. This resource paradigm, has overwhelmingly increased the capabilities of mobile applications, simplifying hardware and enabled collaborateive computing. 

Nevertheless, as we begin to rely more on remote resources we also grow more dependant on the communication delay introduced the intermediate network and by the geographical separation of the \ue and the \dc. Latency sensitive applications such as process controlls, storage, and compute offloading will quickly faulter if subject to a significant and varying delay.

The virtual resources are accessed through increasingly congested mobile access networks. With more devices arecrowding the mobile networks, and aplications are generating and receiving more data, this congestion translates to delay. Additionally, the geographic distance to the data centre introduces a propagation delay.

The \xcloud paradigm, put forward by \cite{chandra2013decentralized,ericsson-akami}, attemps to remedy the aformentioned congestion and delay by locating cloud resources at various stategic nodes in and adjacent to the mobiel access network. At one extreme \dc resources can propsedly be located in at the edge of the network, adjacent or integrated into an \rbs, catering for the \ues reciding in its cells. Alternatively, or complimenarty, \dcs can be integrated with resources in the proposed forthcoming virtualized radio access networks.

The concept of geo-disributed resources has been worked on 

The geograpghic proximity between the \ue and the \dc is an esscential parameter when eliminating application service delay, to that effect, services are migrated with the \ue, through the network to minimize this incurred delay. Services, or rather the VMs that host the service is migrated to the node that is available, provides the lowest delay, and least global network congestion. However, by doing so might minimize the experiance delay for the \ue, but will incurr a migration overhead in \dc and in the network a VM is migrated. Conceivably, various schemes and cost functions can be deployed to minimize both the delay experianced by the user and the added resource strain to th \dc and the network.

User mobility is a fundamental dynamic property of \xcloud, and it is essential to understand how \ue mobility affects the perceived service performance and what load it imposes on the network. 

This paper provides an investigation into the fundamental effects of user \ue in the \xcloud in relation to the number of subscribers, the abstract placement of the servers, and the numer of services. An optimal or reasonable technical bounds for the \xcloud topology is not yet to be determined. This paper disregards the deeper techincal and topological constrains of existing mobile systems in order to provide fundamental resulst that can be employed to shape the forthcoming mobile nework generations.

% benefits drawn from caching in edge networks \cite{1437087}
% related research