\section{Introduction}

Mobile services and \ue{} \footnote{Any user client device accesing the service, such as a mobile phone} functions are at an increasing rate being virtualized and augmented to the cloud. Rich Mobile Applications \cite{March2011618} will soon, more often than not, be seamlessly executed, partially or fully in the cloud. Alongside applications, fundamental \ue{} resources, such as storage and CPU, are being augmented to the cloud. In this resource paradigm, the border between what is being executed locally and remotely is blurred as developers are given more powerful tools to tap into remote ubiquitous generic virtual resources. Additionally, the advent of the internet of things will contribute with vast number of new types of wireless devices, acutuators, and sensors requireing and connecting to remote virtual and augmented resources. 

% Rich mobile applications

As we begin to rely more on remote resources we also grow more suceptible to the communication delay introduced by the the intermediate WAN network and by the geographical separation of the \ue{} and the \dc{} \cite{choi2007analysis}. Latency sensitive applications and cognitive augmentation servcies, such as process controls, latency sensitive storage, real time video game rendering, and augmented reality video analysis will quickly falter if subject to a significant and varying communication delay.

Virtual resources are accessed through increasingly congested mobile access networks. More devices are crowding the mobile networks and applications are generating and receiving more data, this congestion cpntributes to communication latency \cite{hu2005measurement}. Additionally, the geographic discrepancy between the \ue and the \dc introduces a propagation delay, bounded by the speed of light.

The \xcloud{} paradigm, put forward by \cite{chandra2013decentralized,ericsson_akami,satyanarayanan2009case,kiukkonen2010towards,March2011618}, attemps to remedy the aforementioned congestion and latency performance inhibitors by locating cloud resources at the edged of, and adjacent to, the mobile access networks. In the ad-hoc scenario, resources are shared amongst \ues{} where each conneted \ue{} surrenders its available resources to its peers. In its centralized form, \dc{} resources are proposedly located at the edge of the network, adjacent or integrated into an \rbs{}, catering for the \ues{} located within its cell coverage. Alternatively, or complimentary, \dcs{} are integrated with resources in the nodes of the proposed virtualized radio access networks. The scale and the degree of dispersion can be optimized for each application, given the applications resource tiers and its users mobility behaviour.

Round trip time, is thus arguably proportioanl to the geographic proximity between the \ue{} and the \dc{}. To that effect, services hosted in the \xcloud{} are migrated with the \ue{}, through the network, to minimize this incurred latency and congestion on the adjacent WAN. In practice, services, or rather the VMs that host the services, are migrated to the \dc{} that, is available, provides the lowest service latency, and incurs least global network congestion. Doing so might minimize the experienced service delay for the \ue{}, but will incurr a migration overhead in the hosting \dc{} and in the network over which the VM is migrated or duplicated. Conceivably, various provisioning schemes and cost functions can be deployed to minimize both the delay experienced by the user and the added resource strain to the \dc{} and the network.

\Ue{} mobility is a key differentiator between traditional cloud computing with distant \dcs{} and the \xcloud{}, and is a fundamental dynamic property of a \xcloud{}. In order to be able to optimize the \xcloud topology, it is essential to understand how \ue{} mobility affects the perceived service performance and what load it imposes on the network in the generic case.

The topology paradigms of tomorrows all-IP (Internet Protocol) mobile networks \cite{6144211,5357099} are still to be determined, but one can assume that they will be influenced by the notion of virtualized resources \cite{baroncelli2010network, chowdhury2009network}. Large portions of \rbss{} can proposedly be virtualized and centralized to a common \dc with a localy-bounded service domain, shared by several \rbss{}, leaving the \rbss, in principal, with just the radio interface \cite{melzercloud}. The expanse of the centralization is geographically bounded by propagation delay and signal attenuation, and is resource hampered by the aggregated traffic that passes through the dedicated \dc. There is to our knowledge, very little research exploring future mobile telecom infrasturcture topologies with the \xcloud{} in mind. There is on the other hand, extensive research directed at exploring relevant economic and IT models of how to integrate existing telecom servies to the cloud and how to apply telecom-grade SLAs to existing cloud services \cite{EricssonWhitePaper,6156350, 5357099}.

The concept of geo-distributed cloud resources has received some attention over the past few years, but has had a clear research focus on storage and shared data. The authors of \cite{agarwal2010volley} present a method to geographically migrate shared data resources globally, not only to minimize the distance between the \ue{} and the \dc{}, and thus service latency, but also to globally load-balance the hosting \dcs{} on which the observed service is distributedly hosted. Their results reveal a significant reduction in service latency, inter-\dc{} communication, and contributed WAN congestion. Their proposed control process runs over long time periods and operate on a global scale with relatively geographically static users. Although sharing some fundamental dynamics, albeit at different scales, in contrast, the \xcloud{} paradigm, \ue{} movement is more rapid and passing from one \rbs to another is likely to occur during a service session. Additionally, \xcloud{} virtualized resources are assumed to be universal and do not just include storage and vary in size and capabilities, deployed by the telecom operators and based on local needs and demand.

The field of \xcloud{} has much in common with field of geo-distributed cloud resources, but is dominated by the notions of augmenting \ues{} through virtualizing their resources \cite{6563280} and reducing service response times through geo-cascaded data caching \cite{1437087,ericsson_akami}. As a result, much of the research is concerned with coping with specific dynamics, and do thus not address the generic case of small geo-distributed \dcs, serving a local mobile subscriber populous. There are, to our knowledge, significant research gaps in how cloud services perform when hyper-disperesed and rapidly migrated, additionally, there is little research on how the \xcloud can be accomodated in future network topologies.

This paper investigates the fundamental effects of \ue mobility in the \xcloud by obsering \dc utilization, how much of the \dcs resources is spent on migrating services, and how service quality is affected by migration. Additionally, this paper investigates some possible fundamental resource constraints on the \dcs in the \xcloud. To this effect, this paper contributes with models designed to examine the fundamental and generic resource problems in a \xcloud{} of mobile \ues{}. The models include a generic mobile network, populated with \ues{} subscribing to a number of services, served by a number of locally geo-distributed \dcs{}. The simulation model is subjected to multiple scenarios in constillations of varying number of users, servics, and \dc clusteirng.

The simulated scenarios reveal ...
%This paper provides an investigation into the fundamental effects of \ue{} in the \xcloud{} in relation to the number of subscribers, the abstract placement of the \dc{}, and the number of services. %An optimal or reasonable technical bounds for the \xcloud{} topology is not yet to be determined. This paper disregards the deeper technical and topological constrains of existing mobile systems in order to provide fundamental results that can be employed to shape the forthcoming mobile network generations.

In this paper, Section \ref{sec:desiard_model} details which aspects and abstractions of the \xcloud topology that are included in our experiments. Furthermore, the simulation model is specified in Section \ref{sec:simulation_model}. Section \ref{sec:experiments} details the specifics of the simulation experiments. Lastly, Sections \ref{sec:results} and \ref{sec:conclusions} present the results and the consultations drawn from the experiments.


%What to add 
%Proposed model
