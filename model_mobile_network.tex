\subsection{Mobile access network}
Forthcoming cell plannig practices will aim increase area energy efficieny by favouring smaller cells in urban areas \cite{shahab2013framework,5360741}. The model will therefor employ a small homogenous mobile network composed of $N_{rbs}$ radio base stations equidistantly distributed. The domain which the network serves is populated by a homogenous group \ues, with a uniform service subscription distribution. A \ue{} is handed over between base stations at the point where they cross the cell boundary distinguishing two independent radio base stations defined by the width of the rectangular cells $d_{rbs}$. The mobile access network model does take into account the physical layer, channel provisioning, and cell load balancing. Additionally, the radio access network functions as a mechanism to associate \ues{} with \dcs{}, propagation and system processing delays are thus not modelled.