\subsection{Mobile access network}
Forthcoming cell planing practices aim to increase area energy efficiency by favouring smaller cells in urban areas \cite{shahab2013framework,5360741}. The model will therefore employ a small homogeneous mobile network composed of $N_{rbs}$ equidistantly distributed radio base stations. The domain which the network serves is populated by a homogeneous group of \ac{MD}s, with a uniform service subscription distribution. A \ac{MD} is handed over between base stations at the point where they cross the cell boundary distinguishing two independent radio base stations defined by the width of the rectangular cells $d_{\ac{RBS}s}$. The mobile access network model does not take into account the physical layer, channel provisioning, and cell load balancing. Additionally, the radio access network functions as a mechanism to associate \ac{MD}s with \ac{DC}s propagation and system processing delays are thus not modelled.

The network is populated by $N_{\ac{MD}}$ each subscribing to one of the $N_{ser}$ available services. For the sake of model simplicity ambient users and traffic have not been modelled.