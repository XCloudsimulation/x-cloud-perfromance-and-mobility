\section{Experiments}
The aforementioned model was implemented in Java employing simjava \cite{SimJava} as the event driven framework. With the constituent models implemented as modules into the event driven framework.

\subsection{Service performance and mobility}
In order to reveal the dynamics between the number of users, placement of the \dcs{}, and the number for services, the simulation was split up into 3 dimensions.
To that effect, simulations were performed for a population of \ues{} $N_{ue}$ ranging from 10 to 500 \ues{} at intervals of 10 \ues{}.
%Additionally, for each run of $N_{ue}$ the number of services $N_{ser}$ and the placement of the \dcs{} was varied. 
Additionally, for each run of $N_{ue}$ the number of services $N_{ser}$ was varied between 1 and 5.
Moreover, the number of \dcs{} $N_{dc}$ was changed between 16, 4 and 1, with each \dc{} serving 1, 4 or 16 \rbss{} respectively.
%The network spans 9 radio access nodes, $N_{rbs}$.
The network spans 16 radio access nodes, $N_{rbs}$.

For each above mentioned simulation scenario, service latency is measured in each \ue{}, and VM migration/duplication resources in the \dc{}.

\subsection{VM migration schemes}
For each scenarios the simulation reveals how often the service resides in each \dc{}, how many times one service is started and stopped in each \dc{}, and how how much service latency is incurred by starting and stopping a service throughout the simulation.

\subsection{Global parameters and principals}
The service time for each \dc{} is set independently, proportional to the mean request generation rate over the number of \rbss{} and the number of VMs running, see Equation \ref{eq:service_time}.
%The \dc{} service time for the ith \dc{} $T_{i,vm}$ is set proportional to the mean request generation rate over the number of \rbss{}, and the number of VMs running on the ith \dc{} $N_{i,vm}$, see Equation \ref{eq:service_time}.

Simulation model parameters used in the experiments can be found in Table~\ref{table:simulation_parameters}, likewise the service parameters are declared in Table~\ref{table:traffic_parameters}.

\begin{equation}
\label{eq:service_time}
T_{i,j,vm} = K \cdot N_{i,vm} \cdot \frac{ \bar{\lambda}_{sys} }{N_{rbs}},
\end{equation}
where $i$, $j$, $vm$, $K$, $N_{i,vm}$, $\bar{\lambda}_{sys}$, $N_{rbs}$. 

\begin{table}[tb]
 	\centering
 	
    \begin{tabular}{|l|l|} \hline
    	\textbf{Parameter}    	& \textbf{Value}			\\ \hline
    	$N_{ue}$						& 10--500 (step: 10)	\\ \hline
    	$N_{rbs}$						& 16								\\ \hline
    	$N_{dc}$						& $\{1,4,16\}$				\\ \hline
    	$N_{ser}$						& 1--5							\\ \hline
    	$T_{ser}$						& \\ \hline
    	$T_{sim}$						& \\ \hline
    	$d_{rbs}$						& \\ \hline
    	$T_{net}$						& \\ \hline
        %$T_{sus}$						& \\ \hline
        %$N_{i,sus vm}$				& \\ \hline
        $T_{vm\_init}$				& \\ \hline
        $T_{vm\_transfer}$		& \\ \hline
		$D_{vm\_transfer}$		& \\ \hline
		$T_{vm\_downtime}$	& \\ \hline
		$D_{memory\_pull}$	& \\ \hline
		$N_{memory\_pull}$	& \\ \hline
    \end{tabular}
    
    \caption{Simulation parameter values}
    \label{table:simulation_parameters}
\end{table}

\begin{table}[tb]
	\centering
	
    \begin{tabular}{|l|l|l|}\hline
    	\textbf{Component}  	& \textbf{Distribution} 	& \textbf{Parameters}     \\ \hline
    	$S_f$   & Pareto   				& K=133000 $\alpha$ =1.1  \\ \hline
    	$S_r$   & Pareto    				& K=1000         \\ \hline
    	$D_r$ 	& Weibull    				& $\alpha$ =1.46 $\beta$ =0.382 \\ \hline
    	$D_s$ 	& Pareto     				& K=1 $\alpha$=1.5      \\ \hline
    \end{tabular}
    
    \caption{Service model components}
    \label{table:traffic_parameters}
\end{table}

To ensure statistical accuracy, each simulation run is independently replicated 10 times.