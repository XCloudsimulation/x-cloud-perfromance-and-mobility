\section{Experiments}
The aforementioned model was implemented in Java employing simjava \cite{SimJava} as the event driven framework. With the constituent models implemented as modules into the event driven framework.

\subsection{Service performance and mobility}

In order to reveal the dynamics between the number of users, placement of the data centres, and the number for services, the simulation is split up into 3 dimensions. To that effect, simulations were performed for a population of \ues{} $N_{ue}$ ranging from 10 to 500 \ues{} at intervals of 10 \ues{}. Additionally, for each run of $N_{ue}$ the number of services $N_{ser}$, ranging from 1 to 4, and the placement of the data centres was varied. The network spans 16 radio access nodes, $N_{rbs}$.

For each above mentioned simulation scenario, service latency is measured in each \ue{}, and VM migration/duplication resource usge in the \dc{}. Resource usage in the \dcs{} is expressed as the quotient of the time spent migrating over the total execution time.

\subsection{VM migration schemes}
For each scenarios the simulation reveals how often the service resides in each \dc{}, how many times one service is started and stopped in each \dc{}, and how how much service latency is incurred by starting and stopping a service throughout the simulation.

\subsection{Global parameters and principals}
The \dc{} service time for the ith \dc{} $T_{i,vm}$ is set proportional with the emperically measured constant $K$ to the mean request generation rate over the number of \rbss{}, $\bar{\lambda}_{sys}$, and the number of VMs running on the ith \dc{} $N{i,vm}$, see Equation \ref{eq:service_time}.

As mentioned previously, the mobile network is seen as a point of access, and that we dissregard channel propagation delay as it is not a variable in the \xcloud system as defined in this paper. Nevertheless, in order to appoint a penalty for intra-\dc{} geographic discrepacy, the core network imposes a Weibull distributed ms delay, with parameters $\alpha=0.5, \beta=0.6$, on intra-\dc{} communication, in multiples of the number of hops.

Simulation model parameters used in the experiments can be found in Table \ref{table:simulation_parameters}, likewise the service parameters are declared in Table \ref{table:traffic_parameters}.

\begin{equation}
\label{eq:service_time}
T_{i,j,vm} = K \cdot N_{i,vm} \cdot \frac{ \bar{\lambda}_{sys} }{N_{rbs}}
\end{equation}

\begin{table}[tb]
 	\centering
 	
    \begin{tabular}{|l|l|} \hline
    	\textbf{Parameter}    	& \textbf{Value} \\ \hline
    	$N_{ue}$						& \\ \hline
    	$N_{rbs}$						& \\ \hline
    	$N_{ser}$						& \\ \hline
    	$T_{ser}$						& \\ \hline
    	$T_{sim}$						& \\ \hline
    	$d_{rbs}$						& \\ \hline
    	$T_{net}$						& \\ \hline
        %$T_{sus}$						& \\ \hline
        %$N_{i,sus vm}$				& \\ \hline
        $T_{vm\_init}$				& \\ \hline
        $T_{vm\_transfer}$		& \\ \hline
		$D_{vm\_transfer}$		& \\ \hline
		$T_{vm\_downtime}$	& \\ \hline
		$D_{memory\_pull}$	& \\ \hline
		$N_{memory\_pull}$	& \\ \hline
    \end{tabular}
    
    \caption{Simulation parameter values}
    \label{table:simulation_parameters}
\end{table}

\begin{table}[tb]
	\centering
	
    \begin{tabular}{|l|l|l|}\hline
    	\textbf{Component}  	& \textbf{Distribution} 	& \textbf{Parameters}     \\ \hline
    	$S_f$   & Pareto   				& K=133000 $\alpha$ =1.1  \\ \hline
    	$S_r$   & Pareto    				& K=1000         \\ \hline
    	$D_r$ 	& Weibull    				& $\alpha$ =1.46 $\beta$ =0.382 \\ \hline
    	$D_s$ 	& Pareto     				& K=1 $\alpha$=1.5      \\ \hline
    \end{tabular}
    
    \caption{Service model components}
    \label{table:traffic_parameters}
\end{table}

To ensure statistical accuracy, each simulation run is independently replicated 10 times.