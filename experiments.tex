\section{Experiments}
The aformentioned model was implemented in java employing simJava \cite{SimJava} as the event driven framework. With the constitient models implemented as modules into the event driven framework.

In order to reveal the dynamics between the the number of users, placement of the data centers, and the number for services. The simulation is split up into 3 dimensions. To that effect, simulations were performed for a population of \ues $N_{ue}$ ranging from 10 to 500 \ues at intervals of 10 \ues. Additionally, for each run of $N_{ue}$ the number of services $N_{ser}$ and the placement of the data centers varied. The network spans $N_{rbs}$ 9 radio access nodes.

The \dc service time $T_{service}$ is set proportional to the mean request generation reate over the numer of  \rbss, and the number of VMs running on the ith \dc $N{i,vm}$, see Equation \ref{eq:service_time}.

\begin{equation}
\label{eq:service_time}
T_{i,j,vm} = K \cdot N_{i,vm} \cdot \frac{ \bar{\lambda}_{sys} }{N_{rbs}}
\end{equation}

Each simulation run is idependantly replicated 10 times.